%%
%% This is file `sample-sigconf.tex',
%% generated with the docstrip utility.
%%
%% The original source files were:
%%
%% samples.dtx  (with options: `sigconf')
%% 
%% IMPORTANT NOTICE:
%% 
%% For the copyright see the source file.
%% 
%% Any modified versions of this file must be renamed
%% with new filenames distinct from sample-sigconf.tex.
%% 
%% For distribution of the original source see the terms
%% for copying and modification in the file samples.dtx.
%% 
%% This generated file may be distributed as long as the
%% original source files, as listed above, are part of the
%% same distribution. (The sources need not necessarily be
%% in the same archive or directory.)
%%
%% The first command in your LaTeX source must be the \documentclass command.
\documentclass[sigconf]{acmart}
\usepackage{booktabs}
\usepackage{multirow}
\usepackage{lscape}
\usepackage{graphicx}

%%
%% \BibTeX command to typeset BibTeX logo in the docs
\AtBeginDocument{%
  \providecommand\BibTeX{{%
    \normalfont B\kern-0.5em{\scshape i\kern-0.25em b}\kern-0.8em\TeX}}}

%% Rights management information.  This information is sent to you
%% when you complete the rights form.  These commands have SAMPLE
%% values in them; it is your responsibility as an author to replace
%% the commands and values with those provided to you when you
%% complete the rights form.
% \setcopyright{acmcopyright}
% \copyrightyear{2020}
% \acmYear{2020}
% \acmDOI{10.1145/1122445.1122456}

%% These commands are for a PROCEEDINGS abstract or paper.
\acmConference[UKICER '20]{UKICER '20: United Kingdom and Ireland Computing Education Research Conference}{September 03--04, 2020}{Glasgow, UK}
\acmBooktitle{UKICER '20: United Kingdom and Ireland Computing
  Education Research Conference,
  September 03--04, 2020, Glasgow, UK}
%\acmPrice{15.00}
%\acmISBN{978-1-4503-XXXX-X/XX/XX}


%%
%% Submission ID.
%% Use this when submitting an article to a sponsored event. You'll
%% receive a unique submission ID from the organizers
%% of the event, and this ID should be used as the parameter to this command.
%%\acmSubmissionID{123-A56-BU3}

%%
%% The majority of ACM publications use numbered citations and
%% references.  The command \citestyle{authoryear} switches to the
%% "author year" style.
%%
%% If you are preparing content for an event
%% sponsored by ACM SIGGRAPH, you must use the "author year" style of
%% citations and references.
%% Uncommenting
%% the next command will enable that style.
%%\citestyle{acmauthoryear}

%%
%% end of the preamble, start of the body of the document source.
\begin{document}

%%
%% The "title" command has an optional parameter,
%% allowing the author to define a "short title" to be used in page headers.
\title[COVID-19, ``Emergency Remote Teaching'' and UK Computer Science Education]{The Impact of COVID-19 and ``Emergency Remote Teaching'' on the UK Computer Science Education Community}

%%
%% The "author" command and its associated commands are used to define
%% the authors and their affiliations.
%% Of note is the shared affiliation of the first two authors, and the
%% "authornote" and "authornotemark" commands
%% used to denote shared contribution to the research.

\author{Tom Crick}
\orcid{0000-0001-5196-9389}
\affiliation{%
  \institution{Swansea University}
  \city{Swansea}
  \country{UK}
}
\email{thomas.crick@swansea.ac.uk}

\author{Cathryn Knight}
\orcid{0000-0002-7574-3090}
\affiliation{%
  \institution{Swansea University}
  \city{Swansea}
  \country{UK}
}
  \email{cathryn.knight@swansea.ac.uk}

\author{Richard Watermeyer}
\orcid{0000-0002-2365-3771}
\affiliation{%
  \institution{University of Bristol}
  \city{Bristol}
  \country{UK}
}
\email{richard.watermeyer@bristol.ac.uk}

\author{Janet Goodall}
\orcid{0000-0002-0172-2035}
\affiliation{%
  \institution{Swansea University}
  \city{Swansea}
  \country{UK}
}
  \email{j.s.goodall@swansea.ac.uk}


%%
%% By default, the full list of authors will be used in the page
%% headers. Often, this list is too long, and will overlap
%% other information printed in the page headers. This command allows
%% the author to define a more concise list
%% of authors' names for this purpose.
\renewcommand{\shortauthors}{Crick, et al.}

%%
%% The abstract is a short summary of the work to be presented in the
%% article.
\begin{abstract}
The COVID-19 (SARS-CoV-2) pandemic has imposed ``emergency remote
teaching'' across education globally, leading to the closure of
institutions across a variety of settings, from early-years through to
higher education. This paper looks specifically at the impact of these
changes of those teaching in the discipline of computer sciences and
cognate domains. Drawing on the quantitative and qualitative findings
from a UK survey of the educational workforce (N=2,197) conducted in
the immediate aftermath of institutional closures in March 2020 and
the shift to online delivery, we report how those teaching computer
science in various UK settings (n=214) show significantly more
positive attitudes towards the move to online learning, teaching and
assessment than those working in other disciplines; these perceptions
were consistent across schools, colleges and higher education
institutions. However, whilst practitioners noted the opportunities of
these changes for their respective sector -- especially a renewed
focus on the importance of digital skills -- they raised a number of
generalisable concerns on the impact of this shift to online on their
roles, their institutions and their sectors as a whole; for example,
the impact on workload, effective pedagogy and job fragility. More
specifically for computer science practitioners, curricula and
qualifications, there were significant concerns raised regarding the
ability to meaningfully deliver certain core topics such as
programming and collaborative group projects, the impact on various
types of assessment, as well as the practical reduction in teaching
time in the compulsory education sector set aside for so-called
``non-core'' subjects. Based on the data obtained from our rapid
response survey, we offer informed commentary, evaluation and
recommendations for emerging learning and teaching policy and practice
in computer science as we move into the 2020-2021 academic year.
\end{abstract}

%%
%% The code below is generated by the tool at http://dl.acm.org/ccs.cfm.
%% Please copy and paste the code instead of the example below.
%%
% \begin{CCSXML}
% <ccs2012>
%  <concept>
%   <concept_id>10010520.10010553.10010562</concept_id>
%   <concept_desc>Computer systems organization~Embedded systems</concept_desc>
%   <concept_significance>500</concept_significance>
%  </concept>
%  <concept>
%   <concept_id>10010520.10010575.10010755</concept_id>
%   <concept_desc>Computer systems organization~Redundancy</concept_desc>
%   <concept_significance>300</concept_significance>
%  </concept>
%  <concept>
%   <concept_id>10010520.10010553.10010554</concept_id>
%   <concept_desc>Computer systems organization~Robotics</concept_desc>
%   <concept_significance>100</concept_significance>
%  </concept>
%  <concept>
%   <concept_id>10003033.10003083.10003095</concept_id>
%   <concept_desc>Networks~Network reliability</concept_desc>
%   <concept_significance>100</concept_significance>
%  </concept>
% </ccs2012>
% \end{CCSXML}

% \ccsdesc[500]{Computer systems organization~Embedded systems}
% \ccsdesc[300]{Computer systems organization~Redundancy}
% \ccsdesc{Computer systems organization~Robotics}
% \ccsdesc[100]{Networks~Network reliability}

%%
%% Keywords. The author(s) should pick words that accurately describe
%% the work being presented. Separate the keywords with commas.
\keywords{COVID-19, emergency remote teaching, practitioner
perceptions, pedagogy, assessment, curriculum, computer science
education}

%%
%% This command processes the author and affiliation and title
%% information and builds the first part of the formatted document.
\maketitle

\section{Introduction}\label{intro}

The impact of COVID-19 is currently incalculable; it has affected, and
continues to affect, profound social suffering, significant cultural
disruption, and deep economic hardship. While indiscriminate in terms
of whom it infects, it has largely punished society’s most vulnerable
and less fortunate~\cite{vonbraun-et-al:2020,lancetcovid:2020}; worse
now, it appears that the virus must be tolerated on an indefinite
basis~\cite{kissler-et-al:2020}. The impact of the pandemic on the
wider education system, across all settings, has been
profound~\cite{oecdcovidedu:2020,armitage+nellums:2020}

Building on our recent work looking at the impact of COVID-19 on the higher
education sector~\cite{watermeyer-et-al:he2020}...

Also linking back to recent work on CS/STEM curriculum reform, digital
skills, accreditation,
etc~\cite{brown-et-al-toce2014,tryfonas+crick:petra2018,crick-et-al:fie2019,davenport-et-al:educon2020,prickett-et-al:iticse2020}


\begin{itemize}
\item Look at HE paper for some intro context
\item CS specific context: shifts in the discipline, school curriculum
reform, wider digital economy, high-value digital skills across the UK
\item Mention “UK”/devolved nations context but say this is aggregated
– link back to research question
\item Research question: “How do computer scientists in the field of
education perceive the rapid shift to ERT?”
\item Data collected in the Immediate aftermath of the lockdown and
shift to online, recognising this is a limitation, but this links to
planning and preparation for next academic, supporting emerging policy
and practice. i.e. what we can learn from this, how might the
sector/discipline change as a result?
\item Impact on practitioners, institutions and their respective
sectors, with derived/implied impact on learners and students.
\item Define LTA
\end{itemize}

\section{Methods}\label{methods}

\subsection{Sample}

The survey aimed to investigate how the education workforce has viewed
the move to online LTA. The target population was those who are
actively involved in the delivery of LTA within the education
sector. Those who did not meet this criterion were excluded from
analysis post-hoc.

We adopted a convenience sampling approach in distributing the survey
whereby a link to the survey was advertised via professional networks
and related education organisations in addition to various social
media channels. After excluding those that did not meet the
participant requirements 2,197 members of the UK education workforce
responded to the survey. This included 1,148 respondents from the HE
(university) sector (52.3\%), 279 respondents from FE (12.7\%) and 569
respondents from schools (25.9\%). 214 participants indicated that
they taught computer sciences. This included 119 from the HE sector
(55.6\%), 24 from FE (11.2\%) and 71 from schools (33.2\%).

The survey was launched on 26 March 2020 and remained open for four
weeks. Due to the distribution method we cannot calculate the response
rate; however, of those who started the survey, 84.9\% completed it.

\subsection{Questionnaire}

On the first page of the questionnaire respondents were informed that
the research was designed to identify their views and experiences of
the move to online LTA in response to COVID-19. The first section of
the questionnaire consisted of demographic questions in order to
determine how participant characteristics impacted key variables. In
order to identify those who are computer scientists, those who
responded that they worked in the HE sector were asked to select their
discipline from a list created using the UK Joint Academic Coding of
Subjects (JACS) codes. Those who worked in schools and FE were firstly
asked if they taught a particular subject. Those that responded that
they did were then asked to select their subject from a list
containing all curriculum subjects taught across the four nations of
the UK.

Demographic questions were followed by Likert and slider-scale
questions exploring respondents' views of the changes. These included
questions about how prepared and confident they felt about the move to
online teaching.

In addition, respondents were asked three open-ended questions in
order to gain their overall insight into the impact of the changes:
``{\emph{Please provide any comments of how the online learning and
teaching changes brought in as a response to COVID-19 will impact
upon}}'' followed by ``{\emph{your role}}'', ``{\emph{your
institution}}'' and ``{\emph{your sector of education}}''.

The survey was piloted on a subsample of the population before
distribution to the wider education workforce.

\subsection{Analysis}

Quantitative bivariate chi-square ($\chi^2$) analysis of the key variables
was conducted in order to determine overall attitudes to online LT\&A
and whether there were significant differences between those in
computer sciences and those in other disciplines.  As there were more
participants from computer sciences that responded from HE
institutions it was necessary to control for the effect of setting on
these outcomes. Furthermore, it could be hypothesised that variables
such as gender and years working in education may have also impacted
the participants responses to these questions. Therefore, binary
logistic regression was used in order to control for these variables.

Qualitative analysis of the open-ended questions used thematic
analysis. Thematic analysis has been described as ``a method for
identifying, analysing and reporting patterns (themes) within
data''~\cite[p.78]{braun+clarke:2006}. This was done by firstly reading
through the qualitative responses and numerically coding the data to
identify whether comments were positive, negative or neutral. The
responses were coded by two researchers to insure inter-rater
reliability. Within these codes potential themes were identified: ``a
theme captures something important about the data in relation to the
research question and represents some level of patterned response or
meaning within the data set''~\cite[p.82]{braun+clarke:2006}. These
themes were reviewed rigorously against the data to ensure that they
were compatible with the data and accurately represented the comments. 

\section{Results}\label{results}

\subsection{Quantitative Data}\label{quantdata}

Figure~\ref{fig:partagree} shows that those who work within the
Computer Sciences discipline were significantly more likely to say
that they felt prepared ($\chi^2$(1)= 22.02, p<0,001), confident
($\chi^2$(1)= 22.98, p<0,001), supported by their institution
($\chi^2$(1)= 4.5, p=0.03), held a good working knowledge of
appropriate technologies ($\chi^2$(1)= 47.75, p<0,001), had access to
appropriate technologies ($\chi^2$(1)= 13.19, p<0,001) and were
confident that their students could access online LT\&A ($\chi^2$(1)=
17.16, p<0,001).

\begin{figure*}
\includegraphics[width=\textwidth]{images/particagree.png}
\caption{Percentage of participants that agree to statements about
  online LT\&A}
\label{fig:partagree}
\end{figure*}

Table~\ref{tab:binregs} shows the results from binary regression on each
statement. This demonstrates that the impact of working within the
computer science discipline remains significant when controlling for
setting, gender, and years teaching. It also shows that those in
schools were significantly more likely to agree with the statements
than those in HE and FE.

\begin{landscape}
 % \footnotesize
\begin{table}[]
\resizebox{\textwidth}{!}{%
  \begin{tabular}{llllllllllllllllllllllllll}
    \toprule
Variable & Category & \multicolumn{4}{l}{I feel prepared to deliver online LT\&A} & \multicolumn{4}{l}{\begin{tabular}[c]{@{}l@{}}I feel confident in my ability to\\facilitate online LT\&A\end{tabular}} & \multicolumn{4}{l}{\begin{tabular}[c]{@{}l@{}}My institution has been supportive in\\facilitating the move to online LT\&A\end{tabular}} & \multicolumn{4}{l}{\begin{tabular}[c]{@{}l@{}}I have a good working knowledge of \\ the technologies that are available to \\ support online LT\&A\end{tabular}} & \multicolumn{4}{l}{\begin{tabular}[c]{@{}l@{}}I can access appropriate technologies\\to support my online LT\&A\end{tabular}} & \multicolumn{4}{l}{\begin{tabular}[c]{@{}l@{}}I am confident that all of my students\\will be able to access the teaching and\\assessment that I make available online\end{tabular}}  \\
\midrule
&  & $\beta$ & SE & p & Odds Ratio & $\beta$ & SE & p & Odds Ratio & $\beta$ & SE & p & Odds Ratio & $\beta$ & SE & p & Odds Ratio & $\beta$ & SE & p & Odds Ratio & $\beta$ & SE & p & Odds Ratio\\
\multirow{2}{*}{CS} & Non CS (ref) &  &  &  &  &  &  &  &  &  &  &  &  &  &  &  &  &  &  \\
 & CS & -0.77 & 0.19 & \textless{}0.001 & 0.46 & -0.92 & 0.22 & \textless{}0.001 & 0.4 & -0.7 & 0.14 & 0.04 & 0.61 & -1.59 & 0.27 & \textless{}0.001 & 0.20 & -0.94 & 0.32 \\\addlinespace
\multirow{2}{*}{Gender} & Male (ref) &  &  &  &  &  &  &  &  &  &  &  &  &  &  &  &  &  &  \\
 & Female & 0.32 & 0.11 & 0.005 & 1.37 & 0.35 & 0.12 & 0.005 & 1.42 & -0.07 & 0.14 & 0.6 & 0.93 & 0.40 & 0.13 & 0.002 & 1.50 & 0.23 & 0.16 \\\addlinespace
\multirow{6}{*}{\begin{tabular}[c]{@{}l@{}}Years\\ working\end{tabular}} & 0-5 (ref) &  &  &  &  &  &  &  &  &  &  &  &  &  &  &  &  &  &  \\
 & 6-10 & -0.12 & 0.19 & 0.53 & 0.89 & -0.29 & 0.20 & 0.15 & 0.75 & 0.12 & 0.24 & 0.61 & 1.13 & -0.01 & 0.20 & 0.98 & 1.00 & 0.22 & 0.26 \\
 & 11-15 & -0.0 & 0.18 & 0.99 & 0.99 & 0.02 & 0.2 & 0.89 & 1.03 & 0.51 & 0.22 & 0.02 & 1.66 & 0.08 & 0.20 & 0.68 & 1.08 & 0.29 & 0.25 \\
 & 16-20 & 0.06 & 0.19 & 0.76 & 1.06 & 0.00 & 0.2 & 0.99 & 1.00 & 0.47 & 0.23 & 0.45 & 1.60 & 0.24 & 0.21 & 0.26 & 1.27 & 0.31 & 0.26 \\
 & 21-25 & 0.25 & 0.19 & 0.18 & 1.29 & 0.20 & 0.20 & 0.31 & 1.22 & 0.32 & 0.25 & 0.19 & 1.38 & 0.47 & 0.21 & 0.02 & 1.60 & 0.73 & 0.25 \\
 & 26+ & 0.39 & 0.19 & 0.04 & 1.48 & 0.36 & 0.20 & 0.08 & 1.43 & 0.31 & 0.24 & 0.19 & 1.37 & 0.65 & 0.22 & 0.003 & 1.91 & 0.20 & 0.28 \\\addlinespace
\multirow{3}{*}{Setting} & School (ref) &  &  &  &  &  &  &  &  &  &  &  &  &  &  &  &  &  &  \\
 & FE & 0.67 & 1.78 & \textless{}0.001 & 1.96 & 0.76 & 0.19 & \textless{}0.001 & 2.14 & 0.61 & 0.24 & 0.01 & 1.84 & 0.85 & 0.20 & \textless{}0.001 & 2.35 & 0.70 & 0.24 \\
 & HE & 1.08 & 0.11 & \textless{}0.001 & 2.94 & 0.94 & 0.14 & \textless{}0.001 & 2.55 & 1.07 & 0.24 & \textless{}0.001 & 2.92 & 1.14 & 0.15 & \textless{}0.01 & 3.11 & 0.78 & 0.19 \\\addlinespace
\multicolumn{2}{l}{Constant} & -1.29 & 0.19 & \textless{}0.001 & 0.28 & -1.55 & 0.20 & \textless{}0.001 & 0.21 & -2.29 & 0.25 & \textless{}0.001 & 0.10 & -1.86 & 0.22 & \textless{}0.001 & 0.16 & -2.65 & 0.27\\
\bottomrule
  \end{tabular}%
}
\caption{Binary regressions on each statement}
\label{tab:binregs}
\end{table}
\end{landscape}

\subsection{Qualitative Data}\label{qualdata}

The qualitative data were coded into positive, negative and neutral
responses. Of the 102 computer scientists that commented on the impact
on their role 23 (22.6\%) were positive, 54 (52.9\%) were negative and
25 (24.5\%) were neutral.  94 computer scientists provided a comment
on the impact on their institution, of these 20 (21.3\%) were
positive, 59 (56.7\%) were negative and 15 (15\%) were
neutral. Finally, 67 computer scientists commented on the impact on
their sector, of these 16 (23.9\%) were positive, 36 (53.7\%) were
negative and 15 (22.4\%) neutral.

Key themes were identified within the responses. These will now be
discussed in relation to computer sciences education.

\subsubsection*{Change as progressive}

\begin{quotation}
``{\emph{If used and set up well, it could be amazing.  Breaking down
barriers to EdTech and embracing technology for a connected student
experience}}'' [school]
\end{quotation}

Computer scientists mentioned a number of progressive and beneficial
aspects to the change to online LT\&A for the discipline. Most
prominently, respondents pointed out how the changes have and would
lead to more recognition of the importance of technology. Common
responses mentioned the ``{\emph{greater staff awareness of
educational technologies}}” [school] and that ``{\emph{everyone
hopefully will now appreciate that digital literacy is important}}''
[school]. This led to many respondents also recognising how computer
science as a subject may have its profile raised by the mass move to
the use of digital technology for learning. One respondent noted
``{\emph{it may put further emphasis on computing as a subject, with
so much technology in use}}'' [school]. As a result, respondents
foresaw long-term benefits for computer sciences as a discipline
within education ``{\emph{ICT has gone up massively as a valued skill
-- hopefully a trend that will be reflected and its impact will be
increased in terms of curriculum timetabling}}'' [school].

Further opportunities were noted in the advance in educational digital
infrastructure. A key theme was the acknowledgement of the
``{\emph{range, quality and resilience of key digital infrastructure
and tools}}'' [HE] and how it may ``{\emph{open new opportunities to
try new online tools}}'' [school]. Furthermore, respondents mentioned
the potential positive impact of financial investment in digital
infrastructure. This was coupled with discussion of the opportunities
for professional development in the area of online LT\&A. It was
recognised that there had been ``{\emph{more ongoing support for staff
with technology}}'' [school] and this would lead to long term
benefits:

\begin{quotation}
``{\emph{It will involve a forced skills upskilling in IT skills for a
number of older members of staff. […] Once over the initial hurdle, I
feel this could be of benefit long term. But this has only happened
due to significant support made available to them to support this}}''
[school]
\end{quotation}

There was also discussion about the positive impact this could have on
pedagogy and practice as it has ``{\emph{opened opportunities for
    flexible learning}}” [school]. Furthermore, respondents mentioned that
``{\emph{it will allow the department to be creative and be innovative
in the way lessons are presented to students}}'' [school]; therefore,
indicating potentially innovation online LT\&A methods for computer
science education.

There was also acknowledgement that while there may be difficulties in
terms of equity of access ``{\emph{computer science will be one of the
least hit as our colleagues and students are among the most capable
when it comes to operating online}}'' [HE].

\subsubsection*{Change as challenge}

\begin{quotation}
``{\emph{I am concerned that my institution thinks a move online is a
move to more innovative and modern teaching, just by virtue of it
being online}}'' [HE]
\end{quotation}

While acknowledgement of opportunities was clear within the
qualitative data, respondents also raised concerns about the impact of
the move to online LT\&A.

A key theme within these concerns was whether the move to online LT\&A
would be as pedagogically beneficial to the students as face to face
teaching. The quote above summarise the concern that top-down messages
from institutions imply that `good practice is online', however, there
is concern that these decisions are not pedagogically-driven. A number
of respondents raised concerns that computer science is
``{\emph{skills-based rather than fact-based.  I long ago abandoned
the traditional lecture as being inappropriate for teaching […] .  Too
often, moving teaching online means moving back to more traditional
teaching styles.}}” [HE] and that ``{\emph{teaching programming
techniques and complex concepts of computer science online is
difficult}}'' [school]. Furthermore, while respondents were aware of
the negative impact of the lack of face-to-face teaching on teaching
computer science, they also acknowledged that ``{\emph{social
interaction is arguably an important component of the student
experience}}'' [HE]. Thus, suggesting wider pedagogical issues due to
the lack of face-to-face teaching.

Along with a perceived negative impact on effective pedagogy due to
the move to online LT\&A, concerns were also raised about the equity
of access to the necessary resources for learning: ``{\emph{online learning in
CS is heavily dependent on pupils' home access}}'' [school]. Furthermore,
concern was raised about the lack of access to labs and computer
science specific software for LT\&A: ``{\emph{access to specialist
laboratories and equipment has been curtailed. Depending upon a
student’s specialism within computer science their experience could be
more significantly affected. For example, those studying networking or
robotics}}'' [HE].

Furthermore, while respondents acknowledged the lack of necessary
physical resource as a problem for their students, many acknowledged
the lack of their own time as a key concern: ``{\emph{the main problem
is not availability of resource and support, but the time needed to
acquire skills in using them, for which there is no space in my
current role}}'' [HE]. The impact of moving resources online on
workload was a concern raised across the discipline: ``{\emph{huge
uncertainty, possibly spending all summer converting a large course to
online without knowing whether/if students will even enrol}}''
[HE]. This concern about the fragility of the sector was particularly
prominent in respondents from HE.

\section{Discussion}\label{discussion}

\begin{itemize}
\item Computer scientists are more prepared and confident and
supported by institution compared to all other disciplines
\item Quant this was significant across sector/gender/etc – the impact
of being a CS is sig predictor of preparedness when controlling for
sector/gender/years teaching, implying CS is sig variable…
\item However, while qual data showed some positive responses there
were a number of concerns around impact of OLTA and their roles going
forward
\end{itemize}

\section{Conclusions and Looking Ahead}\label{conclusions}

\begin{itemize}
\item Looking ahead for the discipline
\item Top down approach to be avoided. Need to recognise disciplinary
uniqueness and pedagogical appropriateness.
\item Emerging sector norms and emerging policy advice/best practice:
WG, UK Gov, QAA, professional body degree accreditation (PSRBs) – also
international comparability;
\item Positive for overall use/impact/perceptions of technology use in
education, with knock-on effects for the discipline of computer
science and its prominence in schools, quals, etc.
\end{itemize}


%%
%% The acknowledgments section is defined using the "acks" environment
%% (and NOT an unnumbered section). This ensures the proper
%% identification of the section in the article metadata, and the
%% consistent spelling of the heading.
% \begin{acks}
% To Robert, for the bagels and explaining CMYK and color spaces.
% \end{acks}

%%
%% The next two lines define the bibliography style to be used, and
%% the bibliography file.
\bibliographystyle{ACM-Reference-Format}
\bibliography{UKICER2020}

\end{document}
\endinput
%%
%% End of file `sample-sigconf.tex'.
